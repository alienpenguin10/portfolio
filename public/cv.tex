\documentclass[11pt,a4paper,sans]{moderncv}        % possible options include font size ('10pt', '11pt' and '12pt'), paper size ('a4paper', 'letterpaper', 'a5paper', 'legalpaper', 'executivepaper' and 'landscape') and font family ('sans' and 'roman')
% modern themes
\moderncvstyle{banking}                            
\moderncvcolor{blue}                               
%\renewcommand{\familydefault}{\sfdefault}         % to set the default font; use '\sfdefault' for the default sans serif font, '\rmdefault' for the default roman one, or any tex font name
\nopagenumbers{}    
\usepackage[utf8]{inputenc}                      
\usepackage[scale=0.88, top=2cm, bottom=0.5cm]{geometry}
%\setlength{\hintscolumnwidth}{3cm}                % if you want to change the width of the column with the dates
%\setlength{\makecvtitlenamewidth}{10cm}           % for the 'classic' style, if you want to force the width allocated to your name and avoid line breaks. be careful though, the length is normally calculated to avoid any overlap with your personal info; use this at your own typographical risks...
\usepackage{import}
\usepackage{multicol}
\setlength{\columnsep}{1cm}
% personal data
\name{Varnie}{Ketheeswaran}
\address{23, Tiggall Close, Earley, Reading, RG6 7ES}{136, Lower Bristol Road, Bath, BA2 3DL}
\phone[mobile]{07311811964}
\email{vk545@bath.ac.uk} 
\homepage{www.github.com/Varnie}
\extrainfo{www.linkedin.com/in/varnie}

% to show numerical labels in the bibliography (default is to show no labels); only useful if you make citations in your resume
%\makeatletter
%\renewcommand*{\bibliographyitemlabel}{\@biblabel{\arabic{enumiv}}}
%\makeatother
%\renewcommand*{\bibliographyitemlabel}{[\arabic{enumiv}]}% CONSIDER REPLACING THE ABOVE BY THIS

% bibliography with mutiple entries
%\usepackage{multibib}
%\newcites{book,misc}{{Books},{Others}}

%----------------------------------------------------------------------------------
%            content
%----------------------------------------------------------------------------------
\begin{document}
%-----       resume       ---------------------------------------------------------
\makecvtitle
\vspace{-\baselineskip}
\vspace{-\baselineskip}
\small{As a highly motivated student with a deep passion for AI and emerging technologies, as well as being the chair of the Cybersecurity Society, I am on the lookout for a placement which will propel my skills and knowledge advancement. My strong drive and enthusiasm, combined with my ability to excel both independently and as part of a team, makes me a valuable asset to any forward-thinking organisation..}

\begin{multicols}{2}
\section{Education}
\subsection{University of Bath, Bath, UK - Computer Science BSc (Hons)}
\textit{Sept 2022 - June 2026}
\begin{itemize}
    \item {
            \textbf{Year 1 }
            \begin{itemize}
                \item \textbf{Artificial  Intelligence 1:} \\
                 AI foundations, problem-solving, logical and probabilistic reasoning, machine learning, modern applications.
                \item \textbf{Principles of programming 1 \& 2 :} \\
                Basics of programming in C, Java and Python, including OOP, complexity and multi-threading
                \item \textbf{Discrete mathematics and databases :} \\
               Introduction to data systems including modelling, storage, access, retrieval, and protection, along with relevant mathematics for computer science.
                \item \textbf{Computer systems architecture 1 \& 2 :} \\
               Understanding computer system architectures, operating systems, and networking.
            \end{itemize} 
        }
    \item {
             \textbf{Year 2}
            \begin{itemize}
                \item \textbf{Artificial  Intelligence 2 :}\\
                Probabilistic reasoning, decision making under uncertainty, reinforcement learning, and logical reasoning using first-order logic.
                
                \item \textbf{Data structures and algorithms:}\\
               Time complexity, Data structures such as lists, stacks, queues, trees, hash tables, and graphs.
                
                \item \textbf{Foundations and frontiers of machine learning :}\\
                Delve into optimization techniques, stochastic gradient descent, backpropagation, neural network architectures, and real-world applications of machine learning.
    
                \item \textbf{Foundations of computation:}\\
                 Explore automata theory and computational complexity, encompassing finite automata and Turing machines, 

                \item \textbf{Machine learning: }\\
                Introduces supervised, unsupervised, and reinforcement learning, including support vector machines, deep neural networks, and Markov Decision Processes.
            \end{itemize} 
        }
\end{itemize}
\subsection{Reading School, Reading, UK - Alevel}
\textit{Sept 2022 - June 2026}
\begin{itemize}
    \item \textbf{Computer Science:} A\textsuperscript{*}
    \item \textbf{Mathematics:} A\textsuperscript{*}
    \item \textbf{Further Mathematics:} A
\end{itemize}

\subsection{Maiden Erlegh School, Reading, UK - GCSE}
\textit{Sept 2022 - June 2026}
\begin{itemize}
    \item \textbf{Grade 9 (A\textsuperscript{*})+:}  Biology, Chemistry,  ICT, Maths, Physics
    \item \textbf{Grade 8 (A\textsuperscript{*}):} Computer Science 
    \item \textbf{Grade 7 (A):} History, Business, English
\end{itemize}


\section{Skills \& Projects}
\subsection{Technical}

\begin{minipage}[t]{0.45\linewidth}
\textbf{Programming Languages}
\begin{itemize}
    \item Python
    \item C
    \item Java
    \item Ruby on Rails
    \item Java Script
\end{itemize}
\end{minipage}
\hfill
\begin{minipage}[t]{0.45\linewidth}
\textbf{Cybersecurity}
\begin{itemize}
    \item Kali Linux, Parrot OS
    \item Wireshark
    \item Metasploit Framework
    \item Burp Suite
    \item John the Ripper
\end{itemize}
\end{minipage}

\begin{minipage}[t]{0.45\linewidth}
\textbf{AI}
\begin{itemize}
    \item Keras
    \item Tensorflow
    \item Scikit learn
    \item Prompt engineering
\end{itemize}
\end{minipage}
\hfill
\begin{minipage}[t]{0.45\linewidth}
\textbf{DevOps and Cloud}
\begin{itemize}
    \item Agile methodologies
    \item Git
    \item Docker
    \item Oracle cloud
\end{itemize}
\end{minipage}


% \subsection{Other}
% \begin{itemize}
%     \item \textbf{Problem-Solving:} Proven record in tackling intricate issues, devising efficient solutions, and timely implementation.
    
%     \item \textbf{Client Relations:} Excelled in client-facing roles, providing superior service and nurturing strong client relationships.
    
%     \item \textbf{Leadership \& Teamwork:} Proficient leader, adept at building motivated teams and collaborating effectively.
    
%     \item \textbf{Effective Communication:} skillful at conveying complex ideas clearly and persuasively.
    
%     \item \textbf{Driving Licence:} Full and clean UK Driving Licence.
% \end{itemize}



\subsection{Notable Projects}
\begin{itemize}
\item{{\textbf{Cybersecurity project:} \textit{'Intrusion Detection System'}\\ 
\small{I used Logistic Regression on KDD Cup 99 data set to develop a model that can accurately detect and classify network attacks in to DoS, Probe, R2L or U2R.}}}
\vspace{6pt}
\item{{\textbf{Deep Learning project:} \textit{'Self Driving Car'}\\ 
\small{Using NVIDIA's research paper on self driving car trained an end-to-end deep learning model that would let a car drive by itself around the track in Udacity driving simulator. It is a supervised regression problem between the car steering angles and the road images in real-time from the cameras of a car. }}}
\vspace{6pt}
\item{{\textbf{Ditchcarbon:} \textit{'Auto Extraction'}\\ 
\small{Leveraged prompt engineering extensively in developing sophisticated software using Ruby on Rails and ChatGPT for efficient data extraction from diverse documents. Automated PDF extraction, processing 200,000 reports. Enhanced accuracy via user-driven prompt and data point refinement. Streamlined data extraction, conversion, and analysis.}}}
\vspace{6pt}
\item{{\textbf{Bath Hackathon:} \textit{'Honest Emissions'} \\
\small{A Chrome extension displays carbon emissions and tree impact for viewed products. It ranks similar products by low CO2 emissions, offers price, rating, and reviews. Empowers informed, eco-conscious shopping. 'Honest Emissions' received recognition as the winner of the 'Best Use of Ditch Carbon API' prize track, further highlighting its impact and effectiveness.}}}
\vspace{6pt}
\item{{\textbf{Local business project:} \textit{'Time Tracker'}\\
\small{Developed distinctive time tracking software with advanced features, such as QR code employee clock-in and optimized staff scheduling algorithms. Resulted in heightened productivity and reduced costs, receiving positive client feedback as an alternative to established solutions like TSheets.}}}
\vspace{6pt}
\end{itemize}

\closesection
% \end{multicols}


% \begin{multicols}{2}
\section{Experience}
\subsection{Chair of Cybersecurity Society, University of Bath}
\textit{September 2023 - Current}\\
Organising events such as CTFs, speakers and workshop for society. I also develop and deliver weekly sessions on Cybersecurity which helps members prepare for the  CTF challenges.
\vspace{5pt}
\subsection{Summer Intern, Ditchcarbon}
\textit{June 2023 - September 2023}\\
Excelled in modern REST API development (Ruby on Rails, B2B), managed, processed, and transformed client data using Python. Created a Carbon Emissions Calculator (React, Typescript, B2C). Developed precise automated PDF data extraction tool (GPT-4) with 95\% accuracy, showcasing prompt engineering proficiency. Managed supplier research pipeline and scoped client requirements adeptly.
\vspace{5pt}
\subsection{Academic Representative, University of Bath}
\textit{October 2022 - Current}\\
Effectively conveyed my peers' viewpoints to staff and faculty, contributing valuable feedback. This role honed my leadership abilities and demonstrated my dedication to academic excellence. Training and support from the university's student union empowered me to enhance the academic experience for all students in the department.
\vspace{5pt}
\subsection{Store Assistant, Rontec Retail}
\textit{August 2022 - June 2023}\\
Worked as a store assistant at a petrol station, collaborating within a proficient team to deliver top-notch customer service. Managed store organization, handled cash transactions meticulously, and ensured inventory restocking. I actively assisted customers, offered product insights, and efficiently resolved any arising concerns.
\vspace{5pt}
\subsection{Coding Tutor, Code Ninjas}
\textit{July 2021 - August 2022}\\
At Code Ninjas, I instructed coding to children aged 7-14, crafting engaging lesson plans and leading coding classes. This experience not only sharpened my teaching skills but also kindled my enthusiasm for technology education.
\vspace{5pt}
\subsection{Head Computing Prefect, Reading School}
\textit{August 2021 - June 2022}\\
During sixth form, I partnered with teachers to shape the curriculum and organized competitions, mentored peers, and promoted computer science within the school. 
\vspace{5pt}
\subsection{Volunteering}
\textit{September 2020 - February 2021}\\
Dedicated my free time to teach math to underprivileged primary school students via the \textit{Future Stories} initiative, creating engaging lesson plans to promote learning and vital math skills. This volunteer work highlights my commitment to community service and passion for inspiring young minds. 
\\

\section{Activities \& Interests}

\begin{itemize}
\item{Starting my journey towards getting a Private Pilot Licence, PPL, by starting to learn the theory.}
\vspace{5pt}

\item{Writing \href{https://cyberblog.notion.site/Cybersecurity-Blog-af4e87264ae0480584d36d45a18eb337?pvs=4}{blog} posts about about Cybersecurity vulnerabilities and attacks.}
\vspace{5pt}

\item{Participating in CTFs and learning new skills in Cybersecurity.}
\vspace{5pt}

\item{As an active member of the Bath Computer Science Society, I engage in coding challenges and capture the flag competitions, significantly boosting my coding skills and problem-solving abilities in software engineering.}
\vspace{5pt}

\item{My love for chess shines in my tournament play and weekly club matches, fostering strategic thinking and enhancing my problem-solving and critical thinking skills. }
\vspace{5pt}

\item{My journey as a novice badminton player highlights the significance of dedication and persistence in skill development, traits that readily apply to software engineering role, emphasizing continuous learning and adaptability. }

\end{itemize}

\section{Qualifications}
\begin{itemize}
\item \textbf{DeepLearning.AI:} Foundation of Deep Learning

\vspace{5pt}

\item \textbf{Kaggle:} Machine Learning

\vspace{5pt}

\item \textbf{Google:} Foundation of Cybersecurity 

\end{itemize}


\section{Achievements}
\begin{itemize}


\item{Created gcsesresource.com, a user-friendly website with extensive GCSE learning materials for GCSE students during COVID-19.}

\vspace{5pt}

\item{Participated and campaigned in Youth Parliament election in my borough, placed 3rd out of 11 candidates.}

\vspace{5pt}

\item{Achieved international rating in Chess.}

\end{itemize}




\section{References}
\subsection{Dr Christopher Clarke}
\textbf{Email:} cjc234@bath.ac.uk \\
\textbf{Phone:} +44 (0) 1225 388993 \\
Personal tutor\\

\subsection{Mr Marc Munier} 
\textbf{Email:} marc@ditchcarbon.com\\
\textbf{Phone:} +44 (0) 1225 384134 \\
CEO of Ditchcarbon \\

\subsection{Mr Adam Dimmick} 
\textbf{Email:} awdimmick@reading-school.co.uk \\
Alevel computer science teacher \\

% Publications from a BibTeX file without multibib
%  for numerical labels: \renewcommand{\bibliographyitemlabel}{\@biblabel{\arabic{enumiv}}}% CONSIDER MERGING WITH PREAMBLE PART
%  to redefine the heading string ("Publications"): \renewcommand{\refname}{Articles}
\nocite{*}
\bibliographystyle{plain}
\bibliography{publications}                        % 'publications' is the name of a BibTeX file

% Publications from a BibTeX file using the multibib package
%\section{Publications}
%\nocitebook{book1,book2}
%\bibliographystylebook{plain}
%\bibliographybook{publications}                   % 'publications' is the name of a BibTeX file
%\nocitemisc{misc1,misc2,misc3}
%\bibliographystylemisc{plain}
%\bibliographymisc{publications}                   % 'publications' is the name of a BibTeX file

%-----       letter       ---------------------------------------------------------
\end{multicols}

\end{document}


%% end of file `template.tex'.
